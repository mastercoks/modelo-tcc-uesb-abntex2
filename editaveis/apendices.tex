% Apêndices
% ---
% Inicia os apêndices
% ---
\begin{apendicesenv} 

% Imprime uma página indicando o início dos apêndices
\partapendices
% ----------------------------------------------------------
\chapter{Questionário sobre à utilização da TI Verde na UESB}
% ----------------------------------------------------------
\textbf{\textit{Expertise} Ambiental}
\begin{enumerate}
    \item Tem conhecimento sobre como diferentes tecnologias computacionais podem funcionar de forma mais eficiente? 
    
    ( \space\space ) Nunca  ( \space\space ) Raramente  ( \space\space ) Algumas vezes  ( \space\space ) Frequentemente ( \space\space ) Sempre
    \item Tem conhecimento sobre as tecnologias computacionais mais limpas e eficientes existentes no mercado? 
    
    ( \space\space ) Nunca  ( \space\space ) Raramente  ( \space\space ) Algumas vezes  ( \space\space ) Frequentemente ( \space\space ) Sempre
    \item Recorre a diferentes fontes para identificar tendências computacionais mais limpas e econômicas (seminários, livros, reportagens, consultorias)? 
    
    ( \space\space ) Nunca  ( \space\space ) Raramente  ( \space\space ) Algumas vezes  ( \space\space ) Frequentemente ( \space\space ) Sempre
    \item Possui um programa de conscientização sobre o uso racional dos recursos computacionais? 
    
    ( \space\space ) Nunca  ( \space\space ) Raramente  ( \space\space ) Algumas vezes  ( \space\space ) Frequentemente ( \space\space ) Sempre
    \item Busca novas formas de redução do consumo de energia dos produtos computacionais (computadores, servidores, datacenters)? 
    
    ( \space\space ) Nunca  ( \space\space ) Raramente  ( \space\space ) Algumas vezes  ( \space\space ) Frequentemente ( \space\space ) Sempre
\end{enumerate}
    
\textbf{Ações Sustentáveis}

\begin{enumerate}
    \item Tem feito suas últimas aquisições tecnológicas levando em consideração a eficiência energética? 
    
    ( \space\space ) Nunca  ( \space\space ) Raramente  ( \space\space ) Algumas vezes  ( \space\space ) Frequentemente ( \space\space ) Sempre
    \item Faz remoção dos equipamentos computacionais que não estão em uso? 
    
    ( \space\space ) Nunca  ( \space\space ) Raramente  ( \space\space ) Algumas vezes  ( \space\space ) Frequentemente ( \space\space ) Sempre
    \item Possui produtos computacionais eficientes em termos de energia? 
    
    ( \space\space ) Nunca  ( \space\space ) Raramente  ( \space\space ) Algumas vezes  ( \space\space ) Frequentemente ( \space\space ) Sempre
    \item Implementa estratégias para melhor utilização dos produtos computacionais (função repouso, refrigeração, área física, virtualização)? 
    
    ( \space\space ) Nunca  ( \space\space ) Raramente  ( \space\space ) Algumas vezes  ( \space\space ) Frequentemente ( \space\space ) Sempre
\end{enumerate}
    
\textbf{Monitoramento}

\begin{enumerate}
    \item Controla os custos com manutenção dos equipamentos computacionais? 
    
    ( \space\space ) Nunca  ( \space\space ) Raramente  ( \space\space ) Algumas vezes  ( \space\space ) Frequentemente ( \space\space ) Sempre
    \item Gerencia o consumo de energia das diferentes tecnologias computacionais? 
    
    ( \space\space ) Nunca  ( \space\space ) Raramente  ( \space\space ) Algumas vezes  ( \space\space ) Frequentemente ( \space\space ) Sempre
    \item Gerencia o desempenho dos equipamentos computacionais? 
    
    ( \space\space ) Nunca  ( \space\space ) Raramente  ( \space\space ) Algumas vezes  ( \space\space ) Frequentemente ( \space\space ) Sempre
\end{enumerate}
    
\textbf{Consciência Socioambiental}

\begin{enumerate}
    \item Possui estratégias e políticas ambientais bem definidas? 
    
    ( \space\space ) Nunca  ( \space\space ) Raramente  ( \space\space ) Algumas vezes  ( \space\space ) Frequentemente ( \space\space ) Sempre
    \item Pode ser considerada ambientalmente sustentável? 
    
    ( \space\space ) Nunca  ( \space\space ) Raramente  ( \space\space ) Algumas vezes  ( \space\space ) Frequentemente ( \space\space ) Sempre
    \item Possui estratégias e políticas para a utilização de recursos naturais (água, luz, papel)? 
    
    ( \space\space ) Nunca  ( \space\space ) Raramente  ( \space\space ) Algumas vezes  ( \space\space ) Frequentemente ( \space\space ) Sempre
    \item Procura parceiros comerciais que têm preocupações ambientais? 
    
    ( \space\space ) Nunca  ( \space\space ) Raramente  ( \space\space ) Algumas vezes  ( \space\space ) Frequentemente ( \space\space ) Sempre
\end{enumerate}
    
\textbf{Orientação Ambiental}

\begin{enumerate}
    \item Faz comunicação constante para apagar a luz ao sair, usar o modo descanso e desligar o computador após o seu uso? 
    
    ( \space\space ) Nunca  ( \space\space ) Raramente  ( \space\space ) Algumas vezes  ( \space\space ) Frequentemente ( \space\space ) Sempre
    \item Faz recomendações aos funcionários de como economizar energia com os produtos computacionais? 
    
    ( \space\space ) Nunca  ( \space\space ) Raramente  ( \space\space ) Algumas vezes  ( \space\space ) Frequentemente ( \space\space ) Sempre
    \item Incentiva a reciclagem de produtos computacionais (ex. papel, cartucho, computador)? 
    
    ( \space\space ) Nunca  ( \space\space ) Raramente  ( \space\space ) Algumas vezes  ( \space\space ) Frequentemente ( \space\space ) Sempre
\end{enumerate}

% ----------------------------------------------------------
\chapter{Roteiro da entrevista realizada na UINFOR sobre a utilização de TI Verde}
% ----------------------------------------------------------

\begin{enumerate}
    \item Quais são as atividades realizadas pela UINFOR?
    \item A UINFOR tem conhecimento do conceito e as práticas da TI Verde? Se sim, quais?
    \item A UINFOR aplica alguma prática de TI Verde? Se sim, quais?
    \item A UINFOR utiliza a prática de Virtualização?
    \subitem Se utilizar:
    \begin{enumerate}
        \item Como é utilizada? O que levou à utilização desta prática? Quais foram os benefícios?
    \end{enumerate}
    \subitem Se não utilizar:
    \begin{enumerate}
        \item Por que não é utilizada?
    \end{enumerate}
    \item A UINFOR utiliza a Gerenciamento Eletrônico de Documentos?
    \subitem Se utilizar:
    \begin{enumerate}
        \item O que levou à utilização desta prática? Quais foram os benefícios?
    \end{enumerate}
    \subitem Se não utilizar:
    \begin{enumerate}
        \item Por que não é utilizada?
    \end{enumerate}
    \item Como é realizado a coleta, armazenamento e descarte dos Resíduos dos Equipamentos Eletroeletrônicos?
    \item Na hora de aquisição de novos equipamentos, quais são os selos levados em conta?
    \item Tem conhecimento dos 5 R’s da Sustentabilidade (Repensar, Recusar, Reduzir, Reutilizar e Reciclar)?
    \item Na sua opinião, como a UESB/UINFOR está hoje em relação a aplicação da TI Verde? O que poderia ser melhorado? E levando tudo em conta, hoje ela pode ser considerada sustentável? Por que?
\end{enumerate}

% ----------------------------------------------------------
\chapter{Roteiro da entrevista realizada sobre o Projeto de Gestão Eletrônica de Documentos (GED)}
% ----------------------------------------------------------

\begin{enumerate}
    \item O que é o projeto do GED?
    \item Quais são os objetivos do projeto?
    \item Qual foi a motivação para implementar o GED?
    \item Quais departamentos onde o projeto foi aplicado?
    \item Quando foi idealizado?
    \item Quais os resultados esperados para quando o GED estiver concluído?
    \item Quais as conquistas do projeto?
    \item Qual é o estado atual do projeto?
    \item Quais as metas ainda a se cumprir?
    \item O que é idealizado para o futuro do projeto?
\end{enumerate}

\end{apendicesenv} 
% ---
