% resumo em português
\setlength{\absparsep}{18pt} % ajusta o espaçamento dos parágrafos do resumo
\begin{resumo}

 Este trabalho apresenta uma análise da utilização da Tecnologia da Informação Verde na Universidade Estadual do Sudoeste da Bahia. Sua produção e publicação partiu da necessidade de conscientização sobre a importância das práticas de TI Verde e da urgência de reduzir os riscos e impactos ambientais causados pela má utilização da tecnologia. Para isso, foi realizado um estudo sobre a TI Verde, descrevendo conceitos, práticas, normas, regulamentações e certificados. Além disso, foi realizada a aplicação de uma entrevista estruturada, uma entrevista semiestruturada e um questionário, a fim de obter um maior conhecimento sobre o objeto de estudo. Com este estudo, espera-se elucidar o dever das organizações quanto a sua responsabilidade social, incentivando-as  a investirem em métodos que tornem suas atividades mais sustentáveis. Através do estudo de caso, foi identificado a falta de estratégias e políticas ambientais e que a instituição implementa algumas práticas verdes, como virtualização e um projeto de Gestão Eletrônica de Documento, mas ainda é necessária a implementação e o aprimoramento de alguns pontos, que são propostos a partir do estudo realizado.

 \textbf{Palavras-chaves}: TI Verde; sustentabilidade; universidade; educação superior.
\end{resumo}