%
% Documento: Introdução
%

\chapter{Introdução}\label{chap:introducao}

Em tempos de globalização, é de suma importância pensar nas gerações futuras. Segundo o relatório de \citeonline{brundtland1987report}, o “desenvolvimento sustentável é o desenvolvimento que satisfaz as necessidades do presente sem comprometer a capacidade das futuras gerações satisfazerem suas próprias necessidades”. 

Para \citeonline{hart1995natural} o desenvolvimento sustentável está relacionado com o crescimento sem prejuízos aos recursos naturais utilizados, onde a preocupação com os fatores ambientais são prioridade. 

Para as empresas e instituições, fica clara a necessidade de se adotar estratégias e modelos para se adequar ao desenvolvimento sustentável, a Tecnologia da Informação Verde.


\section{Justificativa}
Esta pesquisa justifica-se pela possibilidade de conscientizar as demais instituições de ensino e empresas sobre a importância de se implementar modelos de Tecnologia da Informação Verde, além de demonstrar as vantagens, como uma enorme economia de energia, pois já existem dispositivos que, se comparado a um computador convencional, consome apenas 5\% de energia, tendo-se uma economia de 95\%. Além disso, por se tratar de dispositivos pequenos e de fácil manutenção, causará uma redução na geração de lixo eletrônico.

Além disso, com a criação de um protocolo para compras de determinados dispositivos de informática, haverá uma padronização no processo de compra, deixando ele menos burocrático e, com apenas algumas informações, determinar qual será o dispositivo ideal para satisfazer todas as suas necessidades, visando sempre um menor custo. 

\section{Questão da pesquisa}
Nos dias atuais, existe uma necessidade das empresas e instituições pensarem em maneiras de otimizar o uso das tecnologias, de modo que contribuam com sustentabilidade do planeta, garantindo assim a preservação do meio ambiente. Neste contexto, existem as seguintes questões que a pesquisa tentará resolver:
\begin{itemize}
    \item Como a Tecnologia da Informação Verde é utilizada na Universidade Estadual do Sudoeste da Bahia, campus de Vitória da Conquista?
    \item A Universidade Estadual do Sudoeste da Bahia utiliza a Tecnologia da Informação Verde de maneira correta?
    \item Como a Tecnologia da Informação Verde pode ser aplicada na Universidade Estadual do Sudoeste da Bahia, visando ter uma redução no consumo de energia, geração de lixo eletrônico, gasto com manutenção?
\end{itemize}

 
\section{Objetivo geral}
Demonstrar os benefícios de se utilizar práticas da Tecnologia da Informação Verde, contribuindo para a diminuição dos impactos ambientais causados pela má utilização da tecnologia, além da redução dos gastos. 

\section{Objetivos específicos}
Realizar um estudo sobre a TI Verde, verificando as soluções que visam o melhor uso da tecnologia da informação, garantindo à redução dos impactos ambientais.
 
Analisar a utilização da TI Verde na Unidade Organizacional de Informática (Uinfor) da UESB, campus de Vitória da Conquista.  

Propor novas soluções baseadas nas práticas da TI Verde, apresentando os ganhos que a instituição terá por implementar essas soluções.


\section{Metodologia}

\subsection{Tipo de pesquisa}
A modalidade desta pesquisa é de campo e bibliográfica. Sendo que quanto aos objetivos é do tipo exploratória e descritiva e quanto à forma de abordagem é qualitativa, com aplicação de entrevista semiestruturada e de um questionário.

\subsection{Campo de pesquisa}
O estudo de caso foi realizado na Unidade Organizacional de Informática da UESB, localizada no prédio da reitoria, no campus de Vitória da Conquista.

\subsection{Coleta de dados}
A obtenção de dados relacionados ao estudo de caso foi realizada aplicando-se um questionário (Apêndice I) e de uma entrevista semiestruturada (Apêndice II) elaboradas pelo pesquisador. As perguntas do questionário e da entrevista foram formadas a partir do conhecimento obtido pelo referencial teórico, se baseando em alguns trabalhos estudados. Sendo contemplados tópicos considerados de relevância para o tema pelo pesquisador de, a fim de coletar dados no que se refere à utilização da TI verde pela instituição, de modo que se possa medir o grau de utilização com base nas respostas encontradas. 

As informações para o embasamento teórico foram coletadas através de fontes primárias como livros, artigos, monografias e leis sobre o referido tema, e fontes complementares sites da internet. 

Após a coleta de dados, foi identificado e medido à utilização da TI Verde, e proposto novas soluções para serem utilizadas pela instituição.

\subsection{Estrutura de desenvolvimento}
A estrutura do desenvolvimento do projeto de pesquisa foi dividido nas seguintes etapas:

\begin{itemize}
  \item Realizar um estudo sobre a TI Verde;
  \item Pesquisar e conhecer como a TI Verde é utilizada atualmente na UESB, em Vitória da Conquista;
  \item Identificar as maiores necessidades e problemas relacionados à tecnologia da informação que a UESB enfrenta;
  \item Propor soluções baseadas nas práticas da TI Verde;
  \item Demonstrar os ganhos que a UESB terá se utilizar as soluções propostas;
  \item Apresentar uma conclusão sobre a utilização da TI Verde na UESB.
\end{itemize}


\section{Estrutura do trabalho}