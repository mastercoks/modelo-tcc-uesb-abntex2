%
% Documento: Introdução
%

\chapter{Introdução}\label{chap:introducao}

Em tempos de globalização econômica e de comunicação, é de suma importância pensar nas gerações futuras. Segundo o relatório de \citeonline{brundtland1987report}, o ``desenvolvimento sustentável é o desenvolvimento que satisfaz as necessidades do presente sem comprometer a capacidade das futuras gerações satisfazerem suas próprias necessidade''. 

Para \citeonline{hart1995natural}, o desenvolvimento sustentável está relacionado com o crescimento sem prejuízos aos recursos naturais utilizados, onde a preocupação com os fatores ambientais são prioridade. 

Para as organizações, fica clara a necessidade de se adotar estratégias e modelos para se adequar ao desenvolvimento sustentável, a Tecnologia da Informação Verde (TI Verde).

\section{Justificativa}

Hoje a Universidade Estadual do Sudoeste da Bahia (UESB) é a maior instituição de ensino da Mesorregião do Centro-Sul Baiano. Ela possui grande influência na comunidade local, tendo a responsabilidade social de implementar práticas sustentáveis e possuir um Sistema de Gestão Ambiental, tornando-se referência de sustentabilidade no âmbito tecnológico, criando uma iniciativa nesta área que atualmente é pouca explorada. Com isso, existe a chance de se conscientizar os gestores das demais instituições de ensino e empresas, sobre a importância, vantagens e ganhos, de se aplicar práticas verdes. 

\section{Questão da pesquisa}
Nos dias atuais, existe a necessidade das organizações pensarem em maneiras para otimizar o uso das tecnologias, de modo que contribuam com a sustentabilidade e o desenvolvimento sustentável, minimizando os riscos e impactos ambientais e atendendo às dimensões da sustentabilidade (ambiental, social e econômica). Neste contexto, existem as seguintes questões que esta pesquisa tentará resolver:
\begin{itemize}
    \item Como a Tecnologia da Informação Verde é utilizada na Universidade Estadual do Sudoeste da Bahia, campus de Vitória da Conquista?
    \item A Universidade Estadual do Sudoeste da Bahia utiliza a Tecnologia da Informação Verde de maneira correta?
    \item Como a Tecnologia da Informação Verde pode ser aplicada na Universidade Estadual do Sudoeste da Bahia, visando ter uma redução no consumo de energia, na geração de Resíduos de Equipamentos Eletroeletrônicos e no gasto com a manutenção?
\end{itemize}

\section{Objetivo geral}
Realizar estudo de caso sobre as práticas de gestão de tecnologia do setor de informática da UESB, partindo do tripé da sustentabilidade, ou seja, as dimensões ambiental, social e econômica e propor melhorias com boas práticas de TI Verde, a partir dos resultados encontrados.

%Conscientizar a UESB sobre a importância de se utilizar práticas da Tecnologia da Informação Verde, reforçando a responsabilidade social que a instituição tem quanto a diminuição dos riscos ou dos impactos ambientais causados pela má utilização da tecnologia.

%Apresentar os benefícios de utilizar práticas da Tecnologia da Informação Verde, contribuindo para a diminuição dos riscos ou dos impactos ambientais causados pela má utilização da tecnologia, além da redução dos gastos. 

\section{Objetivos específicos}
Realizar um estudo sobre a TI Verde, verificando as soluções que visam o melhor uso da tecnologia da informação, a fim de garantir a redução dos impactos ambientais.
 
Analisar a utilização da TI Verde na Unidade Organizacional de Informática da UESB, campus de Vitória da Conquista.  

Propor novas soluções e melhorias baseadas nas práticas da TI Verde.


\section{Metodologia}

\subsection{Tipo de pesquisa}
A modalidade desta pesquisa é de campo e bibliográfica. Quanto aos objetivos é do tipo exploratória e descritiva, e quanto à forma de abordagem é qualitativa, com aplicação de entrevista estruturada, entrevista semiestruturada e de um questionário.

\subsection{Campo de pesquisa}

A entrevista semiestruturada e o questionário foram aplicadas com o diretor da Unidade Organizacional de Informática da UESB (UINFOR). A entrevista foi realizada no dia 15 de Junho de 2018, utilizando o aplicativo de mensagens instantâneas \textit{WhatsApp Messenger}, já o questionário foi aplicado no dia 17 de Abril de 2018, utilizando o editor de texto online Google Documentos. Também foi realizado a leitura do Roteiro de Diagnóstico Setorial, disponibilizado pelo entrevistado, que descreve todas as funções, atribuições, atividades, estruturas e processos da UINFOR, além do Plano de Desenvolvimento Institucional (PID) 2013-2017$^{[}$\footnote{PID 2013-2017, Disponível em: \url{http://www.uesb.br/pdi/arquivos/PDI_Final.pdf}.  Acessado em 16 de junho de 2018}$^{]}$ que contém diversas informações sobre a UESB.

A entrevista estruturada foi aplicada com o analista e desenvolvedor do projeto de Gestão Eletrônica de Documentos (GED), do Setor de Informações Funcionais (SIF). A entrevista foi realizada no dia 14 de Junho de 2018, utilizando o editor de texto online Google Documentos e o aplicativo de mensagens instantâneas \textit{WhatsApp Messenger}.

\subsection{Coleta de dados}
A obtenção de dados relacionados ao estudo de caso foi realizada mediante aplicação de um questionário (Apêndice A), uma entrevista semiestruturada (Apêndice B) e uma entrevista estruturada (Apêndice C). As perguntas das entrevistas foram elaboradas pelo pesquisador a partir do conhecimento obtido pelo referencial teórico, se baseando em alguns trabalhos estudados. Foram contemplados tópicos considerados de relevância para o tema pelo pesquisador, a fim de coletar dados no que se refere à utilização da TI verde pela instituição, de modo que se possa medir o grau de adequação com base nas respostas encontradas.

O questionário aplicado foi extraído de \citeonline{lunardi2014desenvolvimento}, e serve para avaliar o grau de utilização da TI Verde pelas organizações. Ele é dividido em 5 fatores, apresentados a seguir: 
\begin{itemize}
    \item  \textbf{Consciência socioambiental:} Avalia se a organização está consciente da necessidade de abordar as questões ambientas de forma mais proativa;
    \item \textbf{Ações sustentáveis:} Avalia se a organização implementa iniciativas para tornar os processos o mais sustentável possível;
    \item \textbf{\textit{Expertise} ambiental:} Avalia se a organização se submete a experimentar, atualizar e buscar novas abordagens, informações e conhecimentos a fim de aplicar estratégias sustentáveis na área de Tecnologia da Informação;
    \item \textbf{Monitoramento:} Avalia se a organização gerencia as atividades e medidas de Tecnologia da Informação voltadas à redução do consumo de recursos e dos danos ao meio ambiente;
    \item \textbf{Orientação ambiental:} Avalia se a organização está comprometida com a sustentabilidade e com o suporte às inovações ambientais.
\end{itemize}

As informações para o embasamento teórico foram coletadas através de fontes primárias como livros, artigos, monografias e leis sobre o referido tema, e fontes complementares, como sites da internet. 

Após a coleta de dados, foi identificada e medida à utilização da TI Verde, e propostas novas soluções para serem utilizadas pela instituição.

\subsection{Estrutura de desenvolvimento}
A estrutura de desenvolvimento do projeto de pesquisa foi dividido nas seguintes etapas:

\begin{itemize}
  \item Realizar um estudo sobre a TI Verde;
  \item Pesquisar e conhecer como a TI Verde é utilizada atualmente na UESB, em Vitória da Conquista;
  \item Identificar as maiores necessidades e problemas relacionados à Tecnologia da Informação que a UESB enfrenta;
  \item Propor soluções baseadas nas práticas de TI Verde;
  \item Apresentar uma conclusão sobre a utilização de TI Verde na UESB.
\end{itemize}
