
\chapter{Conclusão e Trabalhos Futuros }\label{chap:conclusão}

\section{Conclusão}

As discussões sobre sustentabilidade estão ampliando seu alcance e chegando em áreas diversas, levando a novas práticas e ampla conscientização, mas nem sempre com a rapidez esperada.

A indústria de equipamentos e dispositivos eletroeletrônicos tem produzido novidades de forma constante, gerando inovação tecnológica em equipamentos e práticas  cotidianas, e ao final do tempo de vida útil destes equipamentos uma quantidade de resíduos considerável.

Neste sentido, espera-se que os utilizadores, em pequena ou larga escala, sejam responsáveis ao utilizar estes recursos, a fim de utilizar corretamente e pelo maior tempo possível todos os recursos tecnológicos disponíveis para sua utilização pessoal ou profissional.

O estudo de caso realizado na UESB, apontou mesmo realizando algumas práticas verdes, como virtualização e Gestão Eletrônica de Documentos, existe uma grande necessidade de se investir em TI Verde, como a criação de estrategias e planos ambientais, implementação de um SGA, criação de uma semana anual para divulgar e discutir a sustentabilidade em todos os âmbitos, entre outras.

Se mostrou importante a criação das propostas apresentadas no presente trabalho, e caso as mesmas forem implementadas, teremos um grande avanço em relação ao que foi analisado, podendo incentivar outras organizações, sendo elas públicas ou privadas, a se tornarem sustentáveis, investindo em gestão ambiental e a preservação do meio ambiente, garantindo um futuro melhor para as gerações futuras.

Para finalizar, a partir do desenvolvimento deste trabalho, notou-se que existe a possibilidade de se ampliar o estudo para outras pesquisas. Estas possibilidades são apresentadas como trabalhos futuros.


%A realização do presente trabalho foi de grande relevância para aumentar os conhecimentos do autor sobre a necessidade de aplicar as práticas de TI Verde. Discutir aspectos relacionado aos impactos ambientais, causados pela má utilização da tecnologia, é de suma importância para a preservação do meio ambiente, garantindo um futuro melhor para as gerações futuras.

%Devido a grande influência que a UESB possui na comunidade local, existe uma preocupação em relação a aplicação da TI Verde, sendo importante identificar se a mesma otimiza o uso das tecnologias, de maneira que contribua com a sustentabilidade e o desenvolvimento sustentável. Com isso é necessário realizar um estudo sobre  TI Verde, para poder analisar a UINFOR, setor responsável pela TI, com intuito de propor melhorias e novas soluções.

%Partindo do objetivo de conscientizar a UESB sobre a importância das práticas sustentáveis na utilização da tecnologia, foi feito um longo estudo para ter um grande embasamento teórico, que serviu para poder avaliar o uso da TI Verde, identificando que a UESB implementa práticas como a Virtualização e o projeto de GED. Foram propostos soluções como a criação de estrategias e planos ambientais, implementação de um SGA, criação de uma semana anual para divulgar e discutir a sustentabilidade em todos os âmbitos, entre outras.


%Com relação às dificuldades para o desenvolvimento do trabalho, o referencial teórico foi a etapa mais difícil do estudo. Pois assunto estudado é bastante amplo, possuindo diversos tópicos. Além disso, foi consultado vários autores e trabalhos diferentes, para um melhor entendimento e compreensão do tema.

%Quanto ao levantamento de dados foi uma etapa tranquila, pelo fato de já possuir uma proximidade e conhecimento dos processos e da estado em que a UESB se encontrava em relação a utilização da TI Verde.

%A situação atual quanto à utilização  da TI Verde pela UESB foi melhor do que a esperado, mesmo assim existe vários pontos que ela precisa se adequar para poder ser considerada uma instituição sustentável.

%As propostas sugeridas foram de fácil implementação e, mesmo com o baixo orçamento da instituição, elas podem ser utilizadas, tendo uma redução nos gastos a curto prazo. O dinheiro economizado poderia então ser investido em em outros projetos.

%Finalizando, o presente trabalho se demonstra de grande importância e de grande contribuição para que a UESB se torne uma organização sustentável, cumprindo com a sua responsabilidade social, além de abrir caminho para que as demais organizações, sendo elas públicas ou privadas de qualquer setor, também possam implementar práticas verdes.

\section{Trabalhos futuros}

Como possíveis trabalhos futuros, pode-se apontar:

\begin{itemize}
\item Ampliar o estudo para toda a UESB, analisando também os campi de Jequié e Itapetinga, englobando outros pontos da sustentabilidade, além da TI Verde;
\item Propor um modelo de Tecnologia da Informação Verde e  um Sistema de Gestão Ambienta para serem implementados na UESB;
\item Realizar o mesmo estudo nas demais instituições de ensino superior de Vitória da Conquista, como o Instituto Federal da Bahia (IFBA), Universidade Federal da Bahia (UFBA), Faculdade de Tecnologia e Ciências (FTC) e Faculdade Independente do Nordeste (FAINOR).
\end{itemize}


