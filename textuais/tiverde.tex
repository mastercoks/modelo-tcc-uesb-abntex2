%
% Documento: Tecnologia de Informação Verde
%

\chapter{Tecnologia da Informação Verde}

\section{Informação}

A palavra “informação” tem origem no Latim. É derivada da palavra “informare”. Este termo é composto dos radicais “in”, que significa “em” e “forma”, que pode ser traduzido como “forma” ou “aspecto”. \cite{gramatica2017}.

“A informação depende do contexto (científico, tecnológico, industrial, artístico, cultural, entre outros) e corresponde às aplicações [...] ou transversalidade, qualidade da informação de perpassar todas as áreas. ” \cite{pinheiro2004informaccao}.

Segundo \citeonline[p. 17]{aguilar2009tecnologia}, a informação são grupos dados classificados e organizados, que de alguma forma tenha valor para que uma pessoa ou empresa consiga utilizar para se beneficiar. Na vida real ou empresarial dispor de informação é um fator determinante para o sucesso. O mal-uso e gestão de informações pode trazer prejuízos astronômicos, por exemplo, caso uma grande empresa perca ou tenha divulgada as informações confidenciais de seus clientes. Para ajudar a resolver esses problemas que existe o que chamam de Tecnologia da Informação.

\section{Tecnologia da Informação}

A tecnologia teve participação ativa no desenvolvimento e no progresso da humanidade, foi desenvolvida para satisfazer às necessidades humanas. Ela já existia antes dos homens e dos conhecimentos científicos. Mesmo sem o auxílio da ciência foi capaz de criar estruturas e instrumentos complexos. Seus primeiros passos foram dados junto com nossos ancestrais por meio de inúmeras tentativas que resultaram na fabricação dos primeiros instrumentos e na sabedoria com eles adquirida. \cite{acevedo1998ciencia, veraszto2004projeto}.

A evolução humana tem ligação direta com o avanço da tecnologia, bem como o contrário. Um é inerente ao outro e isso é indiscutível quando observamos as várias mudanças nos instrumentos, que se tornaram mais funcionais, acompanhando a evolução do homem, tanto na anatomia de seu corpo, principalmente de sua mão, quanto em seu cérebro. \cite[p. 107-111]{acevedo1998ciencia}.

A palavra “tecnologia” não tem uma definição precisa. Seu conceito foi interpretado de diferentes maneiras, variando de acordo com a época e o local, baseado em teorias diversas ao longo da história \cite{gama1986}. Conforme a etimologia da palavra, “tecnologia” tem sua origem no grego antigo e deve ser separada em duas partes: téchne, que pode ser definido como técnica, ofício e logia, que significa o estudo de algo. Ou seja, tecnologia é todo o conjunto de conhecimentos práticos ou estudo da técnica. \cite{conceito2017}.

Para \citeonline[p. 2]{de1996tecnologia}, a tecnologia pode ser definida como conjunto de conhecimentos, de sua maioria científicos, que são aplicados em determinados ramos de atividades. Podendo ser considerada uma ciência que trata da técnica.

\begin{citacao}
 Tecnologia é um pacote de informações organizadas, de diferentes tipos (científicas, empíricas...), provenientes de várias fontes (descobertas científicas, patentes, livros, manuais, desenhos...), obtidas através de diferentes métodos (pesquisa, desenvolvimento, cópia, espionagem...), utilizado na produção de bens e serviços. \cite{fleury1990capacitaccao}.
\end{citacao} 

A TI é um dos campos da tecnologia que tomou força no século XX. Segundo \citeonline[p. 2]{de1996tecnologia} “tecnologia da informação pode ser entendida como os meios utilizados pelas empresas produtivas para alavancar e potencializar o processo de criação e desenvolvimento de capacitação tecnológica”.

“A TI aparece como um forte indicador de melhoria na performance e na produtividade organizacional” \cite[p. 2]{lunardi2001efeitos}, além de representar um importante papel na continuação de esforços das empresas para tornarem os seus processos mais ágeis e produtivos \cite{shaw1997information}.

A TI está presente em todos os tipos de empresa e a cada dia torna-se mais indispensável, devido à grande importância do gerenciamento de informações. Infelizmente, nem sempre esse gerenciamento se dá de forma agradável para o meio ambiente \cite[p. 6-7]{silva2011}. É de extrema importância utilizar meios para que a TI não traga graves impactos para as gerações futuras, para solucionar esses problemas, os pesquisadores desenvolveram as práticas da Tecnologia da Informação Verde.


\section{Tecnologia da Informação Verde}

O grande avanço tecnológico dos últimos anos, acompanhado da obsolescência programada dos produtos e do consumo desenfreado com consequente produção em grande escala resultam em grandes riscos para o meio ambiente. O descarte inadequado do lixo eletroeletrônico tem grande impacto ambiental e traz sérios riscos à saúde humana, uma das causas é a contaminação de solos e águas com minérios pesados. Há também o grande gasto de energia, uma vez que o número de aparelhos eletrônicos vem crescendo em empresas e indústrias além de vários serviços que são utilizados 24h horas por dia ou são atualizados sempre, impedindo seu desligamento durante a noite ou fins de semana.

\begin{citacao}
O paradigma atual de desenvolvimento é um modelo meramente capitalista, que visa ao lucro máximo. Portanto, o crescimento econômico em si gera bem-estar à sociedade, e o meio ambiente é apenas um bem privado, no que se refere à produção e descarte dos seus resíduos. Dentro desse processo, ao longo dos últimos 30 anos, pode-se afirmar que os recursos naturais são tratados apenas como matéria-prima para o processo produtivo, principalmente no processo produtivo industrial. O que aconteceu é que este modelo, da maneira como foi idealizado, não é sustentável ao longo do tempo. Ficou claro que os recursos naturais eram esgotáveis, e, portanto, finitos, se mal utilizado. \cite{kraemer2005responsabilidade}.
\end{citacao} 

Segundo \citeonline[p. 2]{laurindo2000estudo}, a tecnologia “não só sustenta as estratégias de negócio existentes, mas também permite que se viabilizem novas estratégias empresariais”. Uma dessas novas estratégias surgiu com a busca pela redução dos impactos ambientais e a possível reversão de danos causados, o que resultou no estabelecimento de práticas sustentáveis na área de TI, que juntas são conhecidas como TI Verde. \cite{aguilar2009tecnologia}.

A Tecnologia da Informação Verde vem do conceito de ecoeficiência, e teve origem na década de 80, e se consiste do uso eficiente dos recursos naturais, e na última década vem sendo utilizado com frequência nos setores de TI. \cite{ferreira2009tiverde}. Atualmente, a TI Verde pode ser definida como um conceito que as empresas de tecnologia criaram para agregar o uso de recursos tecnológicos e políticas que minimizem cada vez mais as agressões ao meio ambiente. \cite{briefing2008preciso}.

\begin{citacao}
A TI Verde já é prática essencial nos departamentos de TI em todo o mundo. E, no Brasil, 51\% dos entrevistados disseram que já possuem estratégias implementadas e 38\% estão começando a falar sobre o assunto. [...] Entre as iniciativas de melhores práticas ambientais, pode ser destacada a substituição de equipamentos antigos com 95\% dos participantes apostando em equipamentos novos e mais eficientes no consumo de energia como parte de suas estratégias verdes, seguido pelo monitoramento do consumo de energia (94\%), virtualização de servidores (94\%) e consolidação de servidores (93\%). Além disso, mais da metade (57\%) dos participantes da pesquisa enxerga o SaaS (Software as a Service) como uma boa alternativa. \cite{ferreira2009tiverde}.
\end{citacao} 

Para \citeonline[p. 36]{aguilar2009tecnologia}, a economia de energia e corte de gasto sempre foram as grandes preocupações das empresas. Essa preocupação se torna ainda maior na área de TI, pois os data centers geralmente costumam figurar como a maior porcentagem dos gastos de energia elétrica de uma companhia, já que em um banco, por exemplo, a energia que a TI utilizada pode superar a metade de todo consumo.  

Além da preocupação com o meio ambiente, várias dessas práticas têm um impacto econômico positivo, o que contribui para sua ampla adoção, já que a busca pelo aumento da produtividade passa pela redução de gastos e tem influência direta da utilização de recursos básicos, como água, energia e matérias primas. Dessa forma, as estratégias da TI Verde são também estratégias de negócio.


\section{Tecnologia da Informação Verde nas instituições e empresas}

A adoção de práticas sustentáveis é bem vista e traz maior reconhecimento para as empresas e instituições que as utilizam, pois, a sociedade atual se preocupa com a preservação do meio ambiente. \cite{abreu2012ti}. Essas práticas são aplicadas de acordo com cada perfil de organização. É necessária uma análise estrutural da empresa para identificar qual prática mais se adequa a sua realidade. A implementação dessas práticas trará benefícios tanto para o meio ambiente, como também para a empresa. \cite{pinto2011estudo}.

\citeonline[p. 7]{pinto2011estudo} classificam as práticas de TI Verde em três níveis: TI Verde de incrementação tática, que não afetam a infraestrutura nem modifica políticas internas já existentes, apenas adiciona medidas como controle do uso de energia, não gerando custo, apenas benefícios; TI Verde Estratégico, que exige mudanças na infraestrutura. É necessário reunir os profissionais de TI para desenvolver novas medidas de produção e utilização tecnologias mais eficientes; TI Verde a Fundo, que integra os níveis anteriores, mas requer mais gastos, visando uma mudança total das instalações, desempenho dos equipamentos e padronização dos processos.

\begin{citacao}
Atualmente as práticas da TI Verde consistem em: Economia de Energia, Virtualização de Servidores e Desktop, Videoconferência, Economia de Papel e Descarte e Reciclagem de Equipamentos Eletrônicos. Com a evolução da tecnologia e suas diferentes formas de utilização, também surgirão novas práticas da TI Verde para que o uso da Tecnologia da Informação seja de forma sustentável. \cite[p. 8]{pinto2011estudo}.
\end{citacao} 

As práticas de TI Verde, nos dias atuais, são utilizadas como estratégias de negócio na maior parte das grandes das empresas. Elas garantem lucros, bem-estar e reconhecimento da empresa, além de ajudar na proteção do meio ambiente, melhorando o futuro das próximas gerações, tornando-se imprescindível no dia a dia de qualquer empresa.